L'intérêt de ce projet est la reproduction d'une carte en trois dimensions générée avec le "`bruit de Perlin"'. Ce sujet nous a été attribué en raison de la variété d'aspects techniques qu'il permet de traiter : les rendus en 3D en temps réel, bien sûr, mais aussi l'aspect visuel et graphique ; la partie fonctionnelle (dans le sens où l'utilisateur doit pouvoir effectuer une liste d'actions, qu'il nous revient de gérer à l'intérieur du programme) ; et enfin, et non des moindres, la partie algorithmique (ce point précis sera développé plus en aval au cours de ce document).
En effet, le  Bruit de Perlin permet la génération de matrices de manière procédurale : ainsi, le rendu visuel est différent à chaque lancement du programme. Le bruit de Perlin fait partie des algorithmes les plus célèbres de certaines industries, comme le jeu vidéo. Le but est donc de pouvoir reproduire ce processus, avec la particularité d'un habillage en "`style Lego"'.

Pour le moment, nous avons reconstitué l'algorithme du bruit de Perlin en Javascript, et nous l'avons intégré à un module three.JS afin de reproduire un monde cubique en 3D.
