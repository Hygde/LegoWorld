L'intérêt de ce projet est la reproduction d'une carte en trois dimensions générée avec le "`bruit de Perlin"'. 
Cet algorithme permet en effet la génération de matrices de manière procédurale ; ainsi, le rendu visuel est différent à chaque lancement du programme. Le bruit de Perlin fait partie des algorithmes les plus célèbres de certaines industries, comme le jeu vidéo. Le but est donc de pouvoir reproduire ce processus, avec la particularité d'un habillage en "`style Lego"'.

Pour le moment, nous avons reconstitué l'algorithme du bruit de Perlin en Javascript, et nous l'avons intégré à un module three.JS afin de reproduire un monde cubique en 3D aux allures de Minecraft.
