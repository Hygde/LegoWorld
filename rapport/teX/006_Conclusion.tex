En guise de conclusion, on peut dire que ce projet aura été instructif sur X points :
\vspace{0.5cm}
\begin{enumerate}
	\item \textbf{Le LateX} : ce langage représente encore une petite difficulté puisque la conception (et non la rédaction) de ce rapport aura été semé d'embûches. Mais nous aurons tout de même appris à nous servir de cette technologie, ce qui s'avèrera sans doute utile plus tard dans le cadre de nos futurs travaux de recherche ou de nos projets professionnels.
	\item \textbf{Three.JS} : cette API demeure un outil puissant, parfois un peu complexe d'utilisation au premier abord, mais riche en paramètres et en possibilités. Si nous ne prétendons pas maîtriser Three.JS à l'issue de ce projet, nous nous contentons tout de même d'être capables de reproduire des environnements WebGL avec un peu d'efforts et d'investissement pour explorer la très riche documentation de cette API.
	\item \textbf{Le bruit de Perlin} : cet algorithme, ou du moins sa version améliorée, demeure au coeur de nombreux travaux aujourd'hui, dans la recherche ou les grandes industries de divertissemement. Il était donc intéressant d'avoir une vue de l'intérieur et d'étudier le fonctionnement de cette algorithme, que nous recroiserons probablement au cours de notre parcours informatique.
\end{enumerate}

La découverte des technologies tel que le webGL, et la rédaction de document en lateX a été très enrichissante. En effet, nous avons pu rapidement voir ce qu'il était possible de faire en webGL à travers l' API three.JS qui propose un large panel d'exemple. De plus, nous avons pu étudier le fonctionnement de l'algorithme de génération de bruit de Perlin qui est très utilisé dans le monde de génération procédurale de terrain. Cependant, il nous reste encore beaucoup à faire pour terminer ce projet.

