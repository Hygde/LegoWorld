La découverte des technologies tel que le webGL, et la rédaction de document en lateX a été très enrichissante. En effet, nous avons pu rapidement voir ce qu'il était possible de faire en webGL à travers l' API three.JS qui propose un large panel d'exemple. De plus, nous avons pu étudier le fonctionnement de l'algorithme de génération de bruit de Perlin qui est très utilisé dans le monde de génération procédurale de terrain. Cependant, il nous reste encore beaucoup à faire pour terminer ce projet.

