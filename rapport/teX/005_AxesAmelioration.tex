Une fois parvenus au terme de cette V1, nous avons donc réalisé la plupart des améliorations que nous avions prévues. Voici cependant celles que nous n'avons pas pu faire :

\vspace{0.5cm}
\begin{enumerate}
	\item \textbf{Rendre le personnage visible} : L'intérêt du projet est également de représenter un personnage évoluant dans l'univers généré. Son apparence reste encore à décider.
	\item \textbf{Implémenter génération de terrain avec le bruit de Simplex }: Le bruit de Simplex étant la version évoluée et plus moderne du bruit de Perlin, il serait intéressant de se pencher dessus afin de saisir les différences entre ces deux algorithmes, et d'observer le résultat sur notre carte en 3D. 
	\item \textbf{Ajouter divers éléments de décors }(arbres, cours d'eau, etc.) : Y ajouter des éléments ponctuels lui donnerait un peu plus de vie et de réalisme.
\end{enumerate}






